\documentclass[12pt]{article}
\usepackage{unicode-math}

\setmainfont[Ligatures=TeX]{TeX Gyre Pagella}
%%\setmathfont{TeX Gyre Pagella Math}
\setmathfont{texgyrepagella-math.otf}
\begin{document}

The xelatex engine allows for Unicode to be entererd directly into the
source file.  This may make your source file more readable.

This is some math text entered with math in the source:
\[
∀X [ ∅ ∉ X ⇒ ∃f:X ⟶  ⋃ X\ ∀A ∈ X (f(A) ∈ A ) ]\]

This is some math text entered with regular markup

\[
\forall X [\emptyset \not\in X \Rightarrow \exists f:X \rightarrow  \bigcup X\ 
    \forall A \in X (f(A) \in A ) ]\]

$∫_ξ^θ f(x)\,dx$


%% Examples taken from 
%% https://tex.stackexchange.com/questions/118244/what-is-the-difference-between-unicode-math-and-mathspec

\end{document}